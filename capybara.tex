% Created 2016-03-14 Mo 12:47
\documentclass[a4paper, captions=tableheading]{tufte-handout}
        \usepackage[utf8]{inputenc}
\usepackage[ngerman, english]{babel}
\usepackage[T1]{fontenc}
\usepackage{footnote}
\usepackage{minitoc}
\usepackage{booktabs}
\usepackage{longtable}
\usepackage{lmodern}
\usepackage{graphicx}
\usepackage{hyperref}
\usepackage{url}
\usepackage{fancyvrb}
\usepackage{color}
\usepackage{xcolor}
\usepackage{amsmath}
\usepackage{amssymb}
\usepackage{array}
\usepackage{listings}
\usepackage{rotating}
\usepackage{multicol}
\usepackage{pdflscape}
\usepackage{ctable}
\usepackage{parskip}
\usepackage{anysize}
\usepackage{supertabular}
\usepackage{minted}
\usepackage{natbib}
\usemintedstyle{perldoc}
\usepackage{gensymb}
\usepackage{nicefrac}
\usepackage{units}
\usepackage{marginfix}
\usepackage{breakurl}
\usepackage{float}
\usepackage{placeins}
\usepackage{tabu}
\usepackage{tabulary}
\usepackage{tocloft}
\usepackage{titlesec}
\newcommand{\sectionbreak}{\clearpage}
\makeatletter
\renewcommand{\@tufte@reset@par}{\setlength{\RaggedRightParindent}{0pt}\setlength{\JustifyingParindent}{0pt}\setlength{\parindent}{0pt}\setlength{\parskip}{0.5\baselineskip}}
\@tufte@reset@par
\renewcommand{\@tufte@margin@par}{\setlength{\RaggedRightParindent}{0pt}\setlength{\JustifyingParindent}{0pt}\setlength{\parindent}{0pt}\setlength{\parskip}{0.5\baselineskip}}
\makeatother
\geometry{left=20mm, textwidth=130mm, marginparsep=8mm, marginparwidth=40mm}
\definecolor{darkblue}{rgb}{0,0,.5}
\definecolor{darkgreen}{rgb}{0,.5,0}
\definecolor{islamicgreen}{rgb}{0.0, 0.56, 0.0}
\definecolor{darkred}{rgb}{0.5,0,0}
\definecolor{mintedbg}{rgb}{0.95,0.95,0.95}
\definecolor{arsenic}{rgb}{0.23, 0.27, 0.29}
\definecolor{prussianblue}{rgb}{0.0, 0.19, 0.33}
\definecolor{coolblack}{rgb}{0.0, 0.18, 0.39}
\definecolor{cobalt}{rgb}{0.0, 0.28, 0.67}
\definecolor{moonstoneblue}{rgb}{0.45, 0.66, 0.76}
\definecolor{aliceblue}{rgb}{0.94, 0.97, 1.0}
\hypersetup{colorlinks=true, breaklinks=true, linkcolor=coolblack, anchorcolor=blue, citecolor=islamicgreen, filecolor=blue,  menucolor=blue,  urlcolor=violet}
\renewcommand\thefootnote{\textcolor{darkred}{\arabic{footnote}}}
\renewcommand{\theFancyVerbLine}{\sffamily\textcolor[rgb]{0.7,0.7,0.7}{\tiny\arabic{FancyVerbLine}}}
\setcounter{secnumdepth}{3}
\titleformat{\section}{\normalfont\Large\itshape\color{black}} {\llap{\colorbox{coolblack}{\parbox{1.5cm}{\hfill\color{white}\thesection}}}}{1em}{}[]
\titleformat{\subsection}{\normalfont\large\itshape\color{black}} {\llap{\colorbox{aliceblue}{\parbox{1.5cm}{\hfill\color{coolblack}\thesubsection}}}}{1em}{}[]
\titleformat{\subsubsection}{\normalfont\large\itshape\color{black}} {\llap{\colorbox{aliceblue}{\parbox{1.5cm}{\hfill\color{coolblack}\thesubsubsection}}}}{1em}{}[]
\author{Göran Kirchner}
\date{\today}
\title{Notes on Ruby}
\hypersetup{
 pdfauthor={Göran Kirchner},
 pdftitle={Notes on Ruby},
 pdfkeywords={},
 pdfsubject={},
 pdfcreator={Emacs 24.5.1 (Org mode 8.3.1)},
 pdflang={English}}
\begin{document}

\maketitle
\tableofcontents


\section{Cucumber}
\label{sec:orgheadline1}

\section{Capybara}
\label{sec:orgheadline11}
\subsection{Navigating}
\label{sec:orgheadline2}
\begin{minted}[bgcolor=mintedbg,frame=none,framesep=0pt,mathescape=true,fontsize=\footnotesize,gobble=0]{ruby}
visit('/projects')
visit(post_comments_path(post))
\end{minted}

\subsection{Clicking links and buttons}
\label{sec:orgheadline3}
\begin{minted}[bgcolor=mintedbg,frame=none,framesep=0pt,mathescape=true,fontsize=\footnotesize,gobble=0]{ruby}
click_link('id-of-link')
click_link('Link Text')
click_button('Save')
click('Link Text') # Click either a link or a button
click('Button Value')
\end{minted}

\subsection{Interacting with forms}
\label{sec:orgheadline4}
\begin{minted}[bgcolor=mintedbg,frame=none,framesep=0pt,mathescape=true,fontsize=\footnotesize,gobble=0]{ruby}
fill_in('First Name', :with => 'John')
fill_in('Password', :with => 'Seekrit')
fill_in('Description', :with => 'Really Long Text_')
choose('A Radio Button')
check('A Checkbox')
uncheck('A Checkbox')
attach_file('Image', '/path/to/image.jpg')
select('Option', :from => 'Select Box')
\end{minted}

\subsection{Scoping}
\label{sec:orgheadline5}
\begin{minted}[bgcolor=mintedbg,frame=none,framesep=0pt,mathescape=true,fontsize=\footnotesize,gobble=0]{ruby}
within("//li[@id='employee']") do
	fill_in 'Name', :with => 'Jimmy'
end

within(:css, "li#employee") do
	fill_in 'Name', :with => 'Jimmy'
end

within_fieldset('Employee') do
	fill_in 'Name', :with => 'Jimmy'
end

within_table('Employee') do
	fill_in 'Name', :with => 'Jimmy'
end
\end{minted}

\subsection{Querying}
\label{sec:orgheadline6}
\begin{minted}[bgcolor=mintedbg,frame=none,framesep=0pt,mathescape=true,fontsize=\footnotesize,gobble=0]{ruby}
page.has_xpath?('//table/tr')
page.has_css?('table tr.foo')
page.has_content?('foo')
page.should have_xpath('//table/tr')
page.should have_css('table tr.foo')
page.should have_content('foo')
page.should have_no_content('foo')
find_field('First Name').value
find_link('Hello').visible?
find_button('Send').click
find('//table/tr').click
locate("//*[@id='overlay'").find("//h1").click
all('a').each { |a| a[:href] }
\end{minted}

\subsection{Scripting}
\label{sec:orgheadline7}
\begin{minted}[bgcolor=mintedbg,frame=none,framesep=0pt,mathescape=true,fontsize=\footnotesize,gobble=0]{ruby}
result = page.evaluate_script('4 + 4');
\end{minted}

\subsection{Debugging}
\label{sec:orgheadline8}
\begin{minted}[bgcolor=mintedbg,frame=none,framesep=0pt,mathescape=true,fontsize=\footnotesize,gobble=0]{ruby}
save_and_open_page
\end{minted}

\subsection{Asynchronous JavaScript}
\label{sec:orgheadline9}
\begin{minted}[bgcolor=mintedbg,frame=none,framesep=0pt,mathescape=true,fontsize=\footnotesize,gobble=0]{ruby}
click_link('foo')
click_link('bar')
page.should have_content('baz')
page.should_not have_xpath('//a')
page.should have_no_xpath('//a')
\end{minted}

\subsection{XPath and CSS}
\label{sec:orgheadline10}
\begin{minted}[bgcolor=mintedbg,frame=none,framesep=0pt,mathescape=true,fontsize=\footnotesize,gobble=0]{ruby}
within(:css, 'ul li') { ... }
find(:css, 'ul li').text
locate(:css, 'input#name').value
Capybara.default_selector = :css
within('ul li') { ... }
find('ul li').text
locate('input#name').value
\end{minted}

\section{RSpec}
\label{sec:orgheadline12}
\end{document}